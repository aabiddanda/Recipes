
\section{Red Meat}
\recipe{Lamb with Grilled Onions}
\recipecolumn{
	\begin{enumerate}
	\item Thinly Sliced Lamb\footnote{This can be found in some Asian grocery stores, particularly for Hot-Pot style dishes};
	\item Salt
	\item Pepper
	\item $\frac{1}{2}$ cup Red Onion (sliced)
	\item 2 Tbsp brown sugar (or Honey)
	\item 1 Tbsp Chili powder 
	\item Butter
	\item 1 Tbsp Lemon Juice
	\end{enumerate}
}{
	\begin{enumerate}	
		\item Unpack the lamb and rub with salt and pepper. Sprinkle with lemon juice and let sit for 5-10 minutes
		\item Mix together sliced onions, brown sugar, and chili powder in bowl
		\item Heat skillet to high and drop small amount of butter into pan. Sear lamb slices on both slides and then reduce heat to medium. Cover and let cook for 1 minute each side 
		\item Place lamb slices on plate, and then place onions in pan to cook
		\item Cook onions until slightly blackened
		\item Place onions on top of lab slices and serve alongside salad\footnote{This dish goes really well with an Argentinian Malbec}
	\end{enumerate}
}

% TODO : Fill in this Recipe! (Revise too) 
\recipe{Dr. Pepper Pork Chops}
\recipecolumn{
	\begin{enumerate}
	\item 2 cans Dr. Pepper
	\item Salt
	\item Pepper
	\item Chili Powder
	\item Cayenne Pepper
	\item Barbecue Sauce
	\item Onion (minced)
	\item Scallions (chopped)
	\item Garlic (minced)
	\item 4 Pork Chops
	\end{enumerate}
}{
	\begin{enumerate}	
	\item In a small pan, place pork chops and marinate with 1 can of Dr.Pepper for approximately 2 hours. 
	\item Drain Dr.Pepper from pan, and dry pork chops. Dust with mixture of chili powder, salt, pepper.
	\item Pour $\frac{1}{2}$ cup of water into pan and cover tightly with foil. Place in oven at 300 degrees for 1 hour
	\item To make the sauce, saute onions and garlic with vegetable oil until soft. Then add barbecue sauce and Dr.Pepper. Let simmer on low/medium heat until sauce consistency is similar to original barbecue sauce. 
	\item Heat pan with vegetable oil to high heat and then place pork chops in pan after removing from oven. After searing on either side, pour in sauce and cook on low heat for 10 minutes. Garnish with chopped scallions.  
	\end{enumerate}
}






\newpage
\section{Poultry}
\recipe{Lemon-Pepper Chicken Salad}
\recipecolumn{
	\begin{itemize}
		\item Chicken (cubed)
		\item Olive Oil
		\item Lemon Pepper Spice Blend
		\item Salad Blend\footnote{I use blend here because it really can work with any blend, I typically use spinach, kale, and chard}
		\item Cherry tomatoes (optional)
		\item Red Onions (optional)
		\item Green Peppers (optional)
		\item Balsamic vinegar
		\item Honey
		\item Feta Cheese
	\end{itemize}
}{
	\begin{enumerate}	
		\item Layer a pan with olive oil and bring to high heat
		\item In a separate bowl, mix together Lemon Pepper mix and chicken cubes by hand
		\item Place chicken in pan and allow to sear for 2 minutes then reduce heat to medium
		\item As chicken cools, place honey and balsamic vinegar in a glass and mix. Microwave for 30 seconds, then place in freezer to cool down for 5 minutes. 
		\item Serve chicken on top of salad greens and veggies, and top off with balsamic dressing and feta cheese. 
	\end{enumerate}
}

\recipe{Moroccan-Style Burgers}
\recipecolumn{
	\begin{itemize}	
		\item 1 Lb Ground Turkey
		\item 1 Medium Onion (diced)
		\item 2 cloves garlic (finely minced)
		\item 1 Tbsp Honey
		\item 2 Tbsp Fresh Parsley
		\item 6 Tbsp Olive Oil
		\item 2 Tbsp Cinnamon
		\item 2 Tbsp Cumin
		\item 1 Tbsp Ginger (dried or fresh minced)
		\item 1 Tbsp Paprika
		\item 1 Tbsp Salt
		\item 1 Tbsp Pepper
	\end{itemize}
}{
	\begin{enumerate}
		\item On medium heat, toast cinnamon, cumin, paprika, salt, and pepper\footnote{Ginger too if it is dried} in dry pan for 2 minutes
		\item Remove dry spices from pan and place in bowl. Mix 2 Tbsp olive oil with lamb and mix.
		\item In pan heat olive oil and sautee onion and garlic with honey\footnote{Ginger if fresh}.\footnote{If a food processor is available, use that to create finer paste of the onions and garlic}\footnote{Adding diced Red Pepper also helps create some flavor}
		\item Remove onion and garlic from pan and pour onto lamb (including oil from pan). Let sit for 5 minutes to lower heat then mix with hands and allow to sit for 5 minutes. 
		\item Heat remaining oil in pan on high heat,  make 1 $\frac{1}{2}$ inch think patties, and sear one side on pan for 2 minutes. 
		\item Reduce heat to medium to cook until desired. Serve with rice and harissa. 
	\end{enumerate}
}	


\recipe{Butter Chicken Tacos}
% Need to supply measurements of everything 
\recipecolumn{
	\begin{itemize}
		\item 1 Chicken Breast Fillet (diced)
		\item Butter Chicken Sauce
		\item Vegetable Oil
		\item Yogurt
		\item Raita (See recipe above)
		\item Onions, Peppers (diced)
		\item Shredded Carrot
		\item Monterrey Jack Cheese
		\item Hard or Soft Taco Shells\footnote{Tostadas work great too!}
	\end{itemize}
}{
	\begin{enumerate}
		\item Heat oil in pan on high heat and cook onions and peppers for 5 minutes
		\item Move vegetables to separate bowl, and place chicken cubes in pan for 5 minutes. Cover in butter chicken sauce and stir. Cover pan and cook on low heat for 5 minutes
		\item While chicken is cooking, mix yogurt, raita masala, and carrot in a bowl
		\item Once chicken is done cooking, serve in hard or soft shell alongside veggies topped with Raita
	\end{enumerate}
}

\newpage
\recipe{Chicken with Citrus Sauce}
\recipecolumn{
	\begin{itemize}
		\item Chicken Breast pieces
		\item Orange Juice (w/ pulp)
		\item Lemon Juice
		\item Lemon Rind
		\item Black Pepper
		\item Honey
		\item Chicken Stock (or Vegetable)
		\item Olive Oil
	\end{itemize}
}{	
	\begin{enumerate}
		\item Heat pan coated with olive oil to medium heat
		\item Cook chicken, slowly adding stock until half stock is remaining
		\item Cover and simmer for 3 minutes
		\item Remove chicken when golden brown and place in separate plate
		\item Add all other ingredients except for black pepper to pan and increase to medium heat
		\item The sauce should be stirred while reducing to a thicker consistency
		\item When sauce is desired consistency, add chicken back into pan and coat in sauce
		\item Sprinkle with black pepper and serve\footnote{A perfect complement to this are green vegetables such as broccoli or stir-fried brussel sprouts}
	\end{enumerate}
}

\recipe{Savory Chicken Cacciatore}
\recipecolumn{
	\begin{itemize}
		\item Chicken Thighs (with skin)
		\item Olive Oil
		\item Brown Sugar
		\item Butter
		\item Red Wine
		\item Vinegar
		\item Canned Diced Tomatoes
		\item Green Peppers
		\item Pepper
		\item Salt 
		\item Brown Sugar
		\item Paprika
		\item Basil
		\item Lemon wedges
	\end{itemize}
}{	
	\begin{enumerate}
		\item Mix together salt, pepper, brown sugar, and paprika to coat chicken in dry rub. Heat large pot to high heat and place chicken in pot.
		\item Once crust begins to form on chicken, lower heat to low and place all other ingredients in pot. Raise temperature again till liquid starts to boil around chicken and then lower to low in order to really let liquid braise the meat.
		\item Garnish with parseley and simple vegetables.
	\end{enumerate}
}



\newpage
\section{Seafood}
\recipe{Pan-Seared Indian-style Swordfish}
\recipecolumn{
	\begin{itemize}
		\item 2 Swordfish steaks (thawed)
		\item Olive oil
		\item Salt
		\item Pepper
		\item Ginger-garlic paste
		\item Turmeric
		\item Garam Masala
		\item Lime Juice\footnote{Lime zest also helps here}
		\item Golden Raisins (crushed or puree)
	\end{itemize}
}{
	\begin{enumerate}
		\item In a small bowl, mix ginger-garlic paste, turmeric, garam masala, lime juice, and raisins.\footnote{If the paste is a little too watery, add a little flour in order to thicken the paste slightly}
		\item Spread spicy paste on both sides of swordfish
		\item In a skillet, heat olive oil on high heat
		\item Once oil is hot, place swordfish steaks and give a nice sear on both sides of the steak (approximately 1 minute on either side)
		\item Reduce heat to medium-low and allow to cook thoroughly
		\item Serve on a bed of basmati rice or alongside roasted vegetables and a nice white wine\footnote{Muscadet wines work particularly well with this, although a dry Riesling or Pinot Gris accomplishes the same effect nicely}
	\end{enumerate}
}

\recipe{Citrus-Vinegar Salmon}
% This is based on a dish found in a previous cookbook called \textit{The Best Simple Recipes}.
\recipecolumn{
	\begin{itemize}
	\item 2 Salmon Fillets (fresh or frozen);
	\item $\frac{1}{8}$ cup Balsamic Vinegar ;
	\item $\frac{1}{8}$ cup Orange Juice;
	\item 1 Tbsp Crushed Red Pepper;
	\item 1 Tbsp Honey;
	\item Olive Oil;
  \item Salt;
	\item Pepper;
	\item Butter;
	\end{itemize}
}{
	\begin{enumerate}
		\item Mix vinegar, honey, Orange Juice, and Red Pepper in a small bowl
		\item Dry off salmon fillets and season with salt and pepper
		\item Heat olive oil in pan on medium-high heat
		\item Place fillets in oil and sear and cover for 10 minutes so entire fillet cooks\footnote{For the sear you want high heat in the pan. After you have a nice golden-brown sear on the fillets, lower the heat to medium-low and cover}
		\item Place fillets onto each serving plate after cooked
		\item In the same skillet, pour in vinegar mixture carefully
		\item Reduce heat to medium-low and let simmer for 3 minutes until sauce reduces in volume
		\item Once sauce is cooled, drizzle over salmon and serve\footnote{This is based on a dish found in a previous cookbook called \textit{The Best Simple Recipes}}
	\end{enumerate}
}

\newpage
\recipe{Citrus Salmon with Lemon Dill Sauce}
% Need measurements
\recipecolumn{
	\begin{itemize}
		\item 2 Salmon Fillets
		\item $\frac{1}{8}$ cup Orange Juice (w/ pulp)
		\item 1 Tbsp Tabasco Sauce
		\item Orange Rind (optional)
		\item Yogurt
		\item Dill Weed
		\item Lemon Juice
	\end{itemize}
}{
	\begin{enumerate}
		\item For the sauce, mix yogurt, lemon-juice, and dill in a serving bowl
		\item For the marinade, mix orange juice and tabasco sauce in a bowl and place salmon fillets in marinade for 5 minutes\footnote{For better absorption of the marinade, lightly poke holes in the fillets with a fork}
		\item While the fillets are marinating, heat oil in a pan at medium-high heat. The oil should be hot so you get a good sear
		\item Remove fillets from marinade, and pat till somewhat dry and season with a sprinkling of salt and pepper
		\item Pan fry the fillets for around 10 minutes until each side becomes golden brown\footnote{Follow same searing technique as previous salmon recipe}
		\item Serve with Lemon-Dill sauce drizzled over the top of each fillet\footnote{Garnish with a curled lemon rind if fanciness is required}
	\end{enumerate}
}

\newpage
\section{Vegetarian}
\recipe{Spicy Quinoa Quesadillas}
\recipecolumn{
	\begin{itemize}
		\item Quinoa
		\item Onion (diced)
		\item Green Pepper (diced)
		\item Black Pepper
		\item Red Chili Pepper (crushed)
		\item Lime Juice
		\item Colby Jack Cheese (shredded)
		\item Tortilla
		\item Yogurt
	\end{itemize}

}{
	\begin{enumerate}
		\item Cook quinoa as instructed on box
		\item Heat oil in separate pan to medium-high and stir fry vegetables
		\item Add lime juice and crushed red pepper to the pan as well\footnote{Instead of lime juice and red pepper, you can also use a chili-lime sauce if you have it}
		\item Add cooked quinoa to the pan and stir in with vegetables and cook for 3 minutes
		\item Heat skillet with a quarter-sized drop of olive oil and lay a tortilla down on the skillet
		\item Sprinkle cheese over the tortilla, then layer with vegetable filling, then layer with cheese and seal with additional tortilla
		\item Flip carefully in the skillet such that cheese on both sides is melted and seals in filling
		\item Serve with a side of salsa and yogurt
	\end{enumerate}
}

\recipe{Stir-Fried Asparagus with White Wine Sauce}
\recipecolumn{
	\begin{itemize}
	\item 10-15 Stalks of Asparagus (washed and cut)
	\item 2 Tbsp Olive Oil
	\item Salt 
	\item Pepper 
	\item Paprika
	\end{itemize}
}{
	\begin{enumerate}
	\item Heat oil in skillet on high
	\item Place asparagus in pan and stir for 5 minutes 
	\item Sprinkle mixture of salt, pepper, and paprika for taste
	\item Remaining on high heat, continue to stir fry the asparagus
	\item Serve drizzled with white wine sauce (see recipe above)
	\end{enumerate}
}

\newpage
% TODO : add measurements to recipe.
\recipe{Okra Stir Fry}
\recipecolumn{
	\begin{itemize}
		\item Okra (cut into $\frac{1}{2}$ inch pieces)
		\item Onion
		\item Cumin
		\item Turmeric
		\item Chili Powder
		\item Garam Masala
		\item Salt
		\item Pepper
		\item Ginger
		\item Garlic
	\end{itemize}
}{
	\begin{enumerate}
		\item Mix together dry spices in a cup. Place onions, ginger, and garlic in food processor and pulse till a fine paste\footnote{Add a teaspoon of olive oil to this paste for a better consistency}
		\item Place small amount of olive oil in pan and fry okra for 5 minutes on high heat. Remove okra and place in separate bowl. 
		\item Add dry spices to hot oil and add onion-based paste and okra back into pan. Cook on medium heat for 5 minutes. 
		\item Turn to high heat for 2 minutes to lightly brown okra and onions.
		\item Serve alongside rice or flatbread.
	\end{enumerate}
}

\recipe{Roasted Root Vegetables with Garlic Butter}
\recipecolumn{
	\begin{itemize}
		\item 2 Sweet Potatoes (diced)
		\item 2 Regular Potatoes (diced)
		\item 2 Onions (diced)\footnote{Add an optional shallot for an extra burst of flavor}
		\item 3 cloves of Garlic (finely chopped)
		\item $\frac{2}{3}$ Tbsp Basil (finely chopped)
		\item $\frac{3}{4}$ stick butter
		\item 2 Tbsp salt
		\item 2 Tbsp pepper
		\item 1 Tbsp Olive Oil
	\end{itemize}
}{
	\begin{enumerate}
		\item Preheat over to 425 degrees
		\item Coat pan with olive oil and place vegetables. Season with salt and pepper and place in oven for 10 minutes. 
		\item While vegetables are going through first roast, put garlic, basil, and butter into bowl and microwave for 40 seconds. Mix butter so it becomes liquid. 
		\item Remove vegetables from oven, drizzle with garlic butter, and place back in oven for 20-25 minutes.
		\item Remove from oven and serve alongside meats or re-fry in pan for homefries. 
	\end{enumerate}
}

\recipe{Honey-Glazed Carrots}
\recipecolumn{
	\begin{itemize}
	\item 5-10 Carrots (Washed and cut into large pieces)
	\item $\frac{1}{2}$ cup vegetable stock
	\item 1 Tbsp Pepper
	\item 1 Tbsp Salt
	\item 1 Tbsp Honey
	\item 1 Tbsp Butter
	\end{itemize}
}{
	\begin{enumerate}
	\item Place stock and carrots in vessel and bring to medium heat and cook for 5-10 minutes.
	\item Add butter, honey, pepper, and salt and turn to high heat. 
	\item Allow stock to reduce to a $\frac{1}{4}$ of the volume then turn off heat and let cool for 10 minutes. 
	\end{enumerate}
}