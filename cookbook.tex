\documentclass[oneside]{recipe}
\usepackage[margin=1in]{geometry}
\usepackage{hyperref}
\usepackage{graphicx}
\usepackage{multicol}
\usepackage{enumitem}

% TODO: make a section for sauces

% Defining custom commands
\newcommand{\recipecolumn}[2]{
	\begin{multicols}{2}
	\raggedcolumns
	#1
	\columnbreak
	#2
	\end{multicols}
}

\begin{document}
\begin{titlepage}
\begin{center}
	\textbf{\Huge The College Cookbook }\\
	\vspace*{\fill}
	\includegraphics[scale=1.75]{college}\\
	\vspace*{\fill}
	\textit{\large by}\\
	\vspace*{5pt}
	\textbf{\large Arjun Biddanda}
	\vspace*{\fill}
	\end{center}
\end{titlepage}

\tableofcontents
\chapter{Foreword}
%Beginning template of a foreword for a 'computer scientist's cookbook'


\chapter{Brunch}
\recipe{Eggs in a Basket}
\recipecolumn{
	\begin{itemize}
		\item Slice of Bread
		\item Egg
		\item Butter
		\item Black Pepper
		\item Green Onion (optional)
	\end{itemize}
}{
	\begin{enumerate}
		\item Heat pan to medium-low, coat with small amount of butter
		\item Use cup to make hole in bread (place bread cutout aside)
		\item Crack egg into hole, and cook 1 minute until egg creates seal with bread
		\item Flip bread slice, allowing other side of the hole to be sealed with egg. Cook desired amount
		\item After cooking garnish with salt and pepper and thinly sliced green onions.
	\end{enumerate}
}

\recipe{Stovetop Granola}
\recipecolumn{
	\begin{itemize}
		\item Butter;
		\item Olive Oil;
		\item Brown Sugar;
		\item Nutmeg;
		\item Oats;
		\item Almonds (chopped);
		\item Dried Cranberries;
		\item Desicated Coconut (optional);
	\end{itemize}
}{
	\begin{enumerate}
		\item Heat up oil and toss oats in the pan on medium heat
		\item Cook oats until they start to brown and crisp. Place oats aside in a bowl and melt butter in same pan
		\item Place brown sugar and nutmeg in pan as well and toss in oats
		\item Let oats get coated in brown sugar mixture, then add other ingredients
		\item After cooking, let cool and store in a container for later
	\end{enumerate}
}

\newpage
\recipe{Sweetened Banana Pancakes}
% Need measurements for filling
\recipecolumn{
	\begin{itemize}
		\item 1 Banana (diced)
		\item Brown Sugar
		\item Canola Oil
		\item Pancake Batter
		\item Salt
	\end{itemize}
}{
	\begin{enumerate}
	\item Heat the oil on medium heat. Place diced banana in pan
	\item Add brown sugar to pan and stir till pudding-like 
	\item Add salt to mixture at end
	\item Stir banana mixture into pancake batter
	\item Make pancakes as normal.
	\end{enumerate}	
}

\recipe{Cornbread Scramble}
\recipecolumn{
	\begin{itemize}
		\item Cornbread (Large Chunks);
		\item Butter;
		\item Salsa;
		\item Eggs;
		\item Milk;
		\item Sausage (or Chorizo);
		\item Cheese;
	\end{itemize}
}{
	\begin{enumerate}
		\item In a bowl, whisk eggs milk and salsa together
		\item Place butter in a pan and allow to melt under medium heat
		\item Cook sausage thoroughly in pan and place into egg mixture
		\item Place cornbread pieces on pan, and after 2 minutes, cover in egg mixture
		\item Periodically scramble eggs in pan so as to mix cornbread and eggs
		\item Turn to low heat and top with cheese and hot sauce (optional) and cover pan for 3 minutes
		\item Serve with Tortillas
	\end{enumerate}
} 

\chapter{Appetizers}
\recipe{Roasted Cauliflower with Almonds}
% Need precise measurements
% Develop sauce to go along with this, like a dijon mustard variant
\recipecolumn{
	\begin{enumerate}
		\item $\frac{1}{2}$ lb caulflower florets (fresh or frozen);
		\item Brown Sugar;
		\item Olive Oil;
		\item Paprika;
		\item Black Pepper;
		\item Sliced Almonds (garnish);
	\end{enumerate}
}{
	\begin{itemize}
		\item Preheat the oven to 450. If the cauliflower florets are frozen, thaw as instructed.
		\item In a bowl, mix olive oil, paprika, and black pepper
		\item Place cauliflower on aluminium foil lined baking pan and drizzle with olive oil mixture
		\item Place florets in oven for 10 minutes or until spots of dark brown begin to evenly appear on cauliflower
		\item Cauliflower should be crispy when taken out of the oven
		\item Serve in a bowl with sliced almonds placed on top as garnish
		\item Serve with Dijon Mustard dipping sauce
	\end{itemize}
}
\newpage
\recipe{Sriracha Seafood Po Boy}
\recipecolumn{
	\begin{itemize}
		\item 2 Frozen Tilapia Fillets
		\item Sriracha Sauce
		\item Lemon Juice
		\item French Bread
		\item Butter
		\item Yogurt
		\item Lettuce (optional)
		\item Tomato (optional)
		\item Onion (optional)
	\end{itemize}
}{
	\begin{enumerate}
		\item Melt butter in a pan on medium heat and add small amounts of lemon juice
		\item When butter is hot, add tilapia fillets to pan and cook for around 5 minutes
		\item Break apart tilapia fillets in the pan and increase heat to high
		\item Add in sriracha sauce to taste and cook until tilapia becomes a little crispier and is coated in Sriracha
		\item Butter two slices of french bread and place tilapia into side bowl
		\item Reduce heat to medium and place bread into same pan
		\item Mix yogurt and a little bit of Sriracha such that it gets a pink color, but not so much of the spice
		\item When the bread is golden brown, take it out and place Sriracha yogurt mix down on either side of bread
		\item Layer sandwich with Tilapia mix as well as lettuce, tomato, and onion if desired
	\end{enumerate}
}

\recipe{Sweet Potato Pancakes}
\recipecolumn{
	\begin{itemize}
		\item 1 Sweet Potato (boiled and mashed)
		\item All-purpose Flour
		\item Black-Pepper
		\item Salt
		\item 1 Egg
		\item Olive Oil
		\item Diced Onion\footnote{Instead of an onion, add one diced or mashed shallot for a more delicate flavor} (optional)
	\end{itemize}
}{
	\begin{enumerate}
		\item Combine the sweet potato, egg, flour, pepper and onion in a bowl and mix well
		\item Take batter and make into balls and then flatten into pancakes.
		\item Leave a little flour in a plate and dredge pancakes
		\item Heat oil in pan under medium heat and fry pancakes in baches till they develop crispy crust (~ 3 minutes)
		\item Take pancakes out of pan and dab with paper towel if there is excess oil on them, then sprinkle with salt and serve\footnote{These pancakes are an excellent complement to the Citrus Salmon with Lemon-Dill Sauce}
	\end{enumerate}
}

\chapter{Sauces}

\recipe{Harissa}
\recipecolumn{
	\begin{itemize}
	\item 1 cup tomato paste
	\item 2 tablespoons ginger-garlic paste
	\item 1 tablespoon red chili flakes
	\item 1 teaspoon salt
	\item 1 teaspoon pepper
	\item 3 tablespoons lemon juice 
	\item $\frac{1}{4}$ cup roasted red pepper (optional)
	\end{itemize}
}{
	\begin{enumerate}
		\item Heat a dry pan to medium-high heat. Then place red chili, salt and pepper in pan and roast for 3 minutes. Allow to cool effectively.
		\item $^*$ If using roasted red pepper, make sure it is blended into a paste first $^*$
		\item In a separate bowl, mix ginger-garlic paste and tomato paste together. Add chili flake mix in and stir till mixed well. 
		\item Add lemon juice to paste and stir. Store in an airtight jar in the refridgerator until serving\footnote{Harissa is a condiment from Northern Africa, typically served with couscous or roasted meats}\footnote{This recipe is adapted from one that I found in a book entitled \textit{Healthy Cooking for people who don't have time to Cook} by Jeanne Jones}
	\end{enumerate}
}

\recipe{Charmoula}
\recipecolumn{
	\begin{itemize}
		\item 1 $\frac{1}{2}$ cups chopped cilantro
		\item 1 $\frac{1}{2}$ cups chopped parsley
		\item 2 tbsp. toasted and ground cumin seeds
		\item 2 cloves garlic
		\item Salt, to taste
		\item Black Pepper, to taste
		\item Turmeric, to taste
		\item $\frac{3}{4}$ Extra-virgin Olive Oil
		\item 2 tbsp. Lemon Juice
	\end{itemize}	
}{
	\begin{enumerate}
		\item blend all ingredients except for lemon juice in a food processor
		\item Add lemon juice and store in fridge till time to be used. Or store in an airtight jar until ready to serve alongside meat or vegetables.  \footnote{Chramoula is a cilantro-based marinade that is from Morocco and typically used for marinating meats or fish. This is a great marinade for salmon or chicken breast, but could be used for grilled vegetable skewers as well}

	\end{enumerate}
}

\recipe{Raita}
\recipecolumn{
	\begin{itemize}	
		\item $\frac{1}{2}$ cup diced\footnote{Or shredded} Cucumber
		\item $\frac{1}{2}$ cup diced tomato (optional)
		\item 1 tablespoon pepper
		\item 2 tablespoons raita masala
		\item 3 cups greek yogurt
		\item 1 cup buttermilk
	\end{itemize}
}{
	\begin{enumerate}
		\item In a blender mix buttermilk, yogurt, raita masala, and pepper
		\item Add cucumber and tomato and store in airtight jar in refridgerator
	\end{enumerate}
}

\chapter{Dinner}

\recipe{Citrus-Vinegar Salmon with Salt Potatoes}
% This is based on a dish found in a previous cookbook called \textit{The Best Simple Recipes}.
\recipecolumn{
	\begin{itemize}
	\item 2 Salmon Fillets (fresh or frozen);
	\item $\frac{1}{2}$ lb Mini Potatoes (yellow or red);
	\item $\frac{1}{8}$ cup Balsamic Vinegar ;
	\item $\frac{1}{8}$ cup Orange Juice;
	\item 1 Tbsp Crushed Red Pepper;
	\item 1 Tbsp Honey;
	\item Olive Oil;
  \item Salt;
	\item Pepper;
	\item Butter;
	\end{itemize}
}{
	\begin{enumerate}
		\item For the potatoes, boil the potatoes in a pot and as they are cooling and water has been removed, toss in butter and coat each potato in butter
		\item Sprinkle salt and pepper over potatoes and toss again
		\item Mix vinegar, honey, Orange Juice, and Red Pepper in a small bowl
		\item Dry off salmon fillets and season with salt and pepper
		\item Heat olive oil in pan on medium-high heat
		\item Place fillets in oil and sear and cover for 10 minutes so entire fillet cooks\footnote{For the sear you want high heat in the pan. After you have a nice golden-brown sear on the fillets, lower the heat to medium-low and cover so that the heat retention in the pan cooks the fillet throughout}
		\item Place fillets onto each serving plate after cooked
		\item In the same skillet, wipe clean and then pour in vinegar mixture carefully
		\item Reduce heat to medium-low and let simmer for 3 minutes until sauce reduces in volume
		\item Add a little bit of butter, salt, and pepper to taste after taking sauce off of heat
		\item Once sauce is cooled, drizzle over salmon and add salt potatoes to plate\footnote{This is based on a dish found in a previous cookbook called \textit{The Best Simple Recipes}}
	\end{enumerate}
}
\newpage

\recipe{Citrus Salmon with Lemon Dill Sauce}
% Need measurements
\recipecolumn{
	\begin{itemize}
		\item 2 Salmon Fillets
		\item $\frac{1}{8}$ cup Orange Juice (w/ pulp)
		\item 1 Tbsp Tabasco Sauce
		\item Orange Rind (optional)
		\item Yogurt
		\item Dill Weed
		\item Lemon Juice
	\end{itemize}
}{
	\begin{enumerate}
		\item For the sauce, mix yogurt, lemon-juice, and dill in a serving bowl
		\item For the marinade, mix orange juice and tabasco sauce in a bowl and place salmon fillets in marinade for 5 minutes\footnote{For better absorption of the marinade, lightly poke holes in the fillets with a fork}
		\item While the fillets are marinating, heat oil in a pan at medium-high heat. The oil should be hot so you get a good sear
		\item Remove fillets from marinade, and pat till somewhat dry and season with a sprinkling of salt and pepper
		\item Pan fry the fillets for around 10 minutes until each side becomes golden brown\footnote{Follow same searing technique as previous salmon recipe}
		\item Serve with Lemon-Dill sauce drizzled over the top of each fillet\footnote{Garnish with a curled lemon rind if fanciness is required}
	\end{enumerate}
}

\recipe{Butter Chicken Tacos}
% Need to supply measurements of everything 
\recipecolumn{
	\begin{itemize}
		\item 1 Chicken Breast Fillet (diced)
		\item Butter Chicken Sauce
		\item Vegetable Oil
		\item Yogurt
		\item Raita Masala\footnote{The instructions to make this spice mix can be found here: \url{http://www.mamtaskitchen.com/recipe_display.php?id=13195}}
		\item Onions, Peppers (diced)
		\item Shredded Carrot
		\item Monterrey Jack Cheese
		\item Hard or Soft Taco Shells
	\end{itemize}
}{
	\begin{enumerate}
		\item Heat oil in pan on high heat and cook onions and peppers for 5 minutes
		\item Move vegetables to separate bowl, and place chicken cubes in pan for 5 minutes. Cover in butter chicken sauce and stir. Cover pan and cook on low heat for 5 minutes. 
		\item While chicken is cooking, mix yogurt, raita masala, and carrot in a bowl. 
		\item Once chicken is done cooking, serve in hard or soft shell alongside veggies topped with yogurt sauce
	\end{enumerate}
}

\newpage
\recipe{Spicy Quinoa Quesadillas}
\recipecolumn{
	\begin{itemize}
		\item Quinoa
		\item Onion (diced)
		\item Green Pepper (diced)
		\item Black Pepper
		\item Red Chili Pepper (crushed)
		\item Lime Juice
		\item Colby Jack Cheese (shredded)
		\item Tortilla
		\item Yogurt
	\end{itemize}

}{
	\begin{enumerate}
		\item Cook quinoa as instructed on box
		\item Heat oil in separate pan to medium-high and stir fry vegetables
		\item Add lime juice and crushed red pepper to the pan as well\footnote{Instead of lime juice and red pepper, you can also use a chili-lime sauce if you have it}
		\item Add cooked quinoa to the pan and stir in with vegetables and cook for 3 minutes
		\item Heat skillet with a quarter-sized drop of olive oil and lay a tortilla down on the skillet.
		\item Sprinkle cheese over the tortilla, then layer with vegetable filling, then layer with cheese and seal with additional tortilla. 
		\item Flip carefully in the skillet such that cheese on both sides is melted and seals in filling. 
		\item Serve with a side of salsa and yogurt
	\end{enumerate}
}

\recipe{Pan-Seared Indian-style Swordfish}
\recipecolumn{
	\begin{itemize}
		\item 2 frozen Swordfish steaks (thawed)
		\item Olive oil
		\item Salt
		\item Pepper
		\item Ginger-garlic paste
		\item Turmeric
		\item Garam Masala
		\item Lime Juice\footnote{Lime zest also helps here}
		\item Golden Raisins (crushed or puree)
	\end{itemize}
}{
	\begin{enumerate}
		\item In a small bowl, mix ginger-garlic paste, turmeric, garam masala, lime juice, and raisins.\footnote{If the paste is a little too watery, add a little flour in order to thicken the paste slightly}
		\item Spread spicy paste on both sides of swordfish.
		\item In a skillet, heat olive oil on high heat. 
		\item Once oil is hot, place swordfish steaks and give a nice sear on both sides of the steak (approximately 1 minute on either side).
		\item Reduce heat to medium-low and allow to cook thoroughly. 
		\item Serve on a bed of basmati rice or alongside roasted vegetables and a nice white wine\footnote{Muscadet wines work particularly well with this, although a off-dry Riesling accomplishes the same effect nicely.}
	\end{enumerate}
}

\newpage
\recipe{Chicken with Citrus Sauce}
\recipecolumn{
	\begin{itemize}
		\item Chicken Breast pieces
		\item Orange Juice (w/ pulp)
		\item Lemon Juice
		\item Lemon Rind
		\item Black Pepper
		\item Honey
		\item Chicken Stock (or Vegetable)
		\item Olive Oil
	\end{itemize}
}{	
	\begin{enumerate}
		\item Heat pan coated with olive oil to medium heat
		\item Cook chicken, slowly adding stock until half stock is remaining
		\item Cover and simmer for 3 minutes
		\item Remove chicken when golden brown and place in separate plate
		\item Add all other ingredients except for black pepper to pan and increase to medium heat
		\item The sauce should be stirred while reducing to a thicker consistency
		\item When sauce is desired consistency, add chicken back into pan and coat in sauce. 
		\item Sprinkle with black pepper and serve\footnote{A perfect complement to this are green vegetables such as broccoli or stir-fried brussel sprouts}
	\end{enumerate}
}

\newpage
\recipe{Easy Vegetable Sabji}
% This recipe is a variation on a recipe found in Charles Mattock's \textit{Eat Cheap but Eat Well} cookbook.
\recipecolumn{
	\begin{itemize}
		\item Cauliflower;
		\item Onion (minced);
		\item Potatoes (peeled, and cubed);
		\item Peas;
		\item Turmeric;
		\item Chili Powder;
		\item Ginger-Garlic paste; 
		\item Salt;
		\item Olive Oil
	\end{itemize}
}{
	\begin{enumerate}
		\item Heat oil in a pan\footnote{Investing in a wok might be worth it for dishes like this} on medium heat and cook onion until tender and translucent
		\item Add turmeric, chili powder, ginger garlic paste, and salt to pan and stir till well blended with onions\footnote{If you would like more gravy as you are cooking, add around $\frac{1}{4}$ cup water here}
		\item Stir in potatoes, cauliflower, and peas and lower heat slightly
		\item Cover and cook till all vegetables are coated in spices and tender\footnote{This recipe is a variation on a recipe found in Charles Mattock's cookbook titled \textit{Eat Cheap but Eat Well}.}
	\end{enumerate}
}

\end{document}