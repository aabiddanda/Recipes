\documentclass[oneside]{recipe}
\usepackage{hyperref}

\begin{document}
\tableofcontents
\chapter{Foreword}

Not a whole lot of inspiration here, just good food that is relatively easy to make. 
\\\\
\textit{- Arjun} 

\chapter{Brunch}
\recipe{Eggs in a Basket}
\ingred{Slice of Bread; 
Egg;
Butter;
Black Pepper;
Green Onion (finely chopped)
}

A timeless classic, this is a simple and easy way to make eggs look really good. Heat the pan to medium-low heat, and coat with light butter. Use a cup to make a hole in the bread, and place this bread cutout aside. Place the bread slice with the hole onto the pan and crack an egg into the hole. After a minute or so, until the egg has formed a seal with the bread. Carefully flip so that the other side of the egg cooks and forms a seal as well (\textit{\textbf{Note: } The flip is the hardest part, it might take a couple times to get it completely right}). Garnish with a sprinkle of black pepper and sliced green onions.

\recipe{Stovetop Granola}
\ingred{
	Butter;
	Olive Oil;
	Brown Sugar;
	Nutmeg;
	Oats;
	Almonds (chopped);
	Dried Cranberries;
	Desicated Coconut (optional);
}

This recipe is very versatile and builds off of the recipe from \href{http://allrecipes.com/recipe/stovetop-granola/}{\underline{here}}. Heat up oil and toss oats in the pan. Cook oats until they start to brown and crisp. Place oats aside in a bowl and melt butter in same pan. Place brown sugar and nutmeg in pan as well and toss in oats. Let oats get coated in brown sugar mixture, and then add other ingredients. After cooking, let cool and store in a container for later. Lots of flexibility in what toppings to add, and custom granola is a great treat on top of regular cereal or ice cream even.

\recipe{Sweetened Banana Pancakes}
% Need measurements for filling
\ingred{1 Banana;
Brown Sugar;
Canola Oil;
Pancake Batter;
Salt;
}

Heat the oil in a small pan on medium heat. Dice the banana finely, and place into the oil. Begin to stir the mixture in the pan until it reaches a pudding-like consistency. 
Continuously sprinkle brown sugar over the top of the mixture while stirring. At the very end, add a little bit of salt to the mixture and stir. Spoon the mixture into pancake batter and make pancakes as normal. \textit{\textbf{Note:} This mixture can also be used as the filling for stuffed french toast}

\recipe{Cornbread Scramble}
\ingred{Cornbread (Large Chunks);
Butter;
Salsa;
Eggs;
Milk;
Sausage (or Chorizo);
Cheese;
}
This is a simple twist on scrambled eggs, but results in a very hearty breakfast. Cornbread is a very versatile dish that you can either buy at the store ready-made or make with the recipe \href{}{\underline{here}}. In a bowl, whisk eggs milk and salsa together. Place butter in a pan and allow to melt under medium heat. Cook sausage thoroughly in pan and place into egg mixture. Again place butter in pan and melt, and then place cornbread pieces on pan. Cover cornbread pieces with egg mixture. Periodically scramble eggs in pan so as to mix cornbread and eggs. Turn to low heat and top with cheese and hot sauce (optional) and cover pan for 3 minutes. Serve while hot with tortillas. 

\recipe{Sweet Potato Pancakes}
\ingred{Sweet Potatoes (sliced into hash brown consistency);
Flour;
Egg;
Onion (finely diced);
Black Pepper
}
% Finish this up later too, and mention to be served with salmon and lemon-dill sauce.


\chapter{Appetizers}
\recipe{Roasted Cauliflower with Almonds}
% Need precise measurements
% Develop sauce to go along with this, like a dijon mustard variant
\ingred{$\frac{1}{2}$ lb caulflower florets (fresh or frozen);
Brown Sugar;
Olive Oil;
Paprika;
Black Pepper;
Sliced Almonds (garnish);
}
Preheat the oven to 450. If the cauliflower florets are frozen, thaw as instructed. In a bowl, mix olive oil, paprika, and black pepper. Place cauliflower on aluminium foil lined baking sheet and drizzle with olive oil mixture. Toss lightly to coat cauliflower evenly. Sprinkle brown sugar over the top of the florets. Place florets in oven for 10 minutes or until spots of dark brown begin to evenly appear on cauliflower. Cauliflower should be crispy when taken out of the oven. Serve in a bowl with sliced almonds placed on top as garnish. Can also be served with Dijon Mustard or Honey Mustard.

\recipe{Sriracha Seafood Po Boy}
\ingred{
	2 Frozen Tilapia Fillets;
	Sriracha Sauce;
	Lemon Juice;
	French Bread;
	Butter;
	Yogurt;
	Lettuce (optional);
	Tomato (optional);
	Onion (optional);
}

Po' Boys are an ideal lunch food or perfect for even an early dinner if you aren't feeling as hungry yet. Thaw the fillets and set aside. Melt butter in a pan on medium heat and add small amounts of lemon juice. When butter is hot, add tilapia fillets to pan and cook for around 5 minutes. Break apart tilapia fillets in the pan and increase heat to high. Add in sriracha sauce to taste and cook until tilapia becomes a little crispier and is coated in Sriracha. Butter two slices of french bread and place tilapia into side bowl. Reduce heat to medium and place bread into same pan (this allows the bread to soak in the flavor as well). Mix yogurt and a little bit of Sriracha such that it gets a pink color, but not so much of the spice. When the bread is golden brown, take it out and place Sriracha yogurt mix down on either side of bread. Then layer sandwich with Tilapia mix as well as lettuce, tomato, and onion if desired. As an alternative, instead of tilapia, shrimp or shredded chicken can be used as well. 

\chapter{Dinner}
\recipe{Citrus Salmon with Lemon Dill Sauce}
% Need measurements
\ingred{2 Salmon Fillets;
$\frac{1}{8}$ cup Orange Juice (w/ pulp);
1 Tbsp Tabasco Sauce;
Orange Rind (optional);
Yogurt;
Dill Weed;
Lemon Juice;
}

For the sauce, mix yogurt, lemon-juice, and dill in a serving bowl. For the marinade, mix orange juice and tabasco sauce in a bowl and placed salmon fillets in marinade for 5 minutes (\textbf{Note:} for better absorption of the marinade, lightly poke holes in the fillets with a fork). While the fillets are marinating, heat oil in a pan at medium heat. Remove fillets from marinade and pat till somewhat dry (a little marinade should remain on the fillets) and season with salt and pepper. Pan fry the salmon fillets for 10 minutes on medium heat, making sure that each side becomes golden-brown. Serve with Lemon-Dill sauce drizzled over the top of each fillet. 

% Finish this recipe 
\recipe{Citrus-Vinegar Salmon with Salt Potatoes}
\ingred{2 Salmon Fillets (fresh or frozen);
$\frac{1}{2} lb$ Mini Potatoes (yellow or red);
$\frac{1}{8}$ cup Balsamic Vinegar ;
$\frac{1}{8}$ cup Orange Juice;
1 Tbsp Crushed Red Pepper;
1 Tbsp Honey;
Olive Oil;
Salt;
Pepper;
Butter;
}

This is based on a dish found in a previous cookbook called \textit{The Best Simple Recipes}. Salt potatoes provide a very nice contrast to pan-seared salmon with a sweet and savory sauce. For the potatoes, boil the potatoes in a pot and as they are cooling and water has been removed, toss in butter and coat each potato in butter. Sprinkle salt and pepper over potatoes and toss again. Set potatoes aside (if they get cool, warm up in microwave before serving). Mix vinegar, honey, Orange Juice, and Red Pepper in a small bowl. Dry off salmon fillets and season with salt and pepper. Heat olive oil in pan on medium-high heat. Place fillets in oil and sear and cover for 5 minutes so entire fillet cooks. Place fillets onto each serving plate after cooked. In the same skillet, wipe clean and then pour in vinegar mixture carefully. Reduce heat to medium-low and let simmer for 3 minutes until sauce reduces in volume. Add a little bit of butter, salt, and pepper to taste after taking sauce off of heat. Once sauce is cooled, drizzle over salmon and add salt potatoes to plate. Adding rice or stir-fried vegetables also complements the salmon quite well. 


\recipe{Butter Chicken Tacos}
% Need to supply measurements of everything 
\ingred{1 Chicken Breast Fillet (diced);
Butter Chicken Sauce;
Vegetable Oil;
Yogurt;
\href{http://www.mamtaskitchen.com/recipe_display.php?id=13195}{Raita Masala};
Onions, Peppers (diced);
Shredded Carrot;
Monterrey Jack Cheese;
Hard or Soft Taco Shells
}

Coat diced chicken cubes in butter chicken sauce, and place into skillet with oil at medium heat for 5 minutes. Add peppers and onions and stir fry for another 5 minutes. Reduce heat and cover mixture with cheese while making sauce.
Mix yogurt, raita masala, and carrot in a bowl. Serve filling in either a hard shell or soft shell topped with yogurt sauce. 

\recipe{Spicy Quinoa Quesadillas}
\ingred{Quinoa;
Onion (diced);
Green Pepper;
Tomato (diced) (optional);
Black Pepper;
Red Chili Pepper (crushed);
Lime Juice;
Colby Jack Cheese (shredded);
Tortilla;
Yogurt
}

Cook the quinoa as instructed, and place aside in separate bowl. Stir fry the vegetables in a separate pan, adding the tomatoes last. Add lime juice and crushed red pepper to the pan as well (\textit{\textbf{Note: } Instead of lime juice and red pepper, you can also use a chili-lime sauce if you have it}) Add the cooked quinoa to the stir fry pan and continue to stir fry for another 3 minutes. Heat skillet with a little olive oil on it and lay a tortilla down on the skillet. Sprinkle cheese over the tortilla, and then layer the quesadilla filling over the cheese. Sprinkle cheese again, and seal with additional tortilla. Flip carefully so that cheese on both sides of filling is melted and seals in filling. Serve alongside yogurt and salsa. 

\recipe{Chicken with Citrus Sauce}
\ingred{Chicken Breast pieces;
Orange Juice (w/ pulp);
Lemon Juice;
Lemon Rind;
Black Pepper;
Honey;
Chicken Stock (or Vegetable);
Olive Oil;
}

Heat pan up and coat with olive oil. Cook chicken on low-medium heat (adding stock slowly until half stock is remaining.) Cover and simmer for 3 minutes. Remove chicken when golden brown and place in separate plate. Add all other ingredients except for black pepper to pan and increase to medium heat. The sauce should be stirred while reducing to a thicker consistency. When happy with the consistency of the sauce, add chicken back into pan and coat with sauce. Sprinkle with black pepper and serve. 

\recipe{Easy Vegetable Sabji}
\ingred{
	Cauliflower;
	Onion (minced);
	Potatoes (peeled, and cubed);
	Peas;
	Turmeric;
	Chili Powder;
	Ginger-Garlic paste; 
	Salt;
	Olive Oil
}
This recipe is a variation on a recipe found in Charles Mattock's \textit{Eat Cheap but Eat Well} cookbook. This is a really good way to incorporate vegetables and spice into your diet, especially if you are not a huge fan of vegetables. Using frozen vegetables is perfectly fine for this recipe and in fact is encouraged since the vegetables last longer that way. 
Heat oil in a pan on medium heat (I prefer a wok) and cook onion until tender and translucent. Add turmeric, chili powder, ginger garlic paste, and salt to pan and stir till well blended with onions (\textbf{Note:} If you would like more gravy as you are cooking, add around $\frac{1}{4}$ cup water here). Stir in potatoes, cauliflower, and peas and lower heat slightly. Cover and cook till all vegetables are coated in spices and tender. 


\end{document}