
% Every good recipe starts with getting groceries. As a student, nobody has time for really long trips to the grocery store every week.

Below, I give a series of helpful tips and tricks that you can use in order to make sure that you are maximizing your efficiency when you cook. 

\begin{enumerate}
\item Plan the week's recipes. This is actually key to making efficient trips to the store. Make sure that there are at least 4 meals on the menu for the week. It can be as simple as ``Make a lot of chicken and store in tupperware'', but it should at least be planned out. 
\item Invest in a single-serving blender. It makes breakfast shakes really easy, and can help make sauces for larger meals as well. 
\item Buy quality meat and fish. This is just a rule since meat and fish are harder to cook and generally students buy less of them. So the little bit that you do buy, make sure that they are the good stuff. 
\item Don't ``overspice''. Spices are expensive and certainly have their place in food. However ``overspicing'' takes away the flavor of the ingredient itself and can really ruin a meal. Top culprits for ``overspicing'' are in grilled meats. 
\item Milk is tough to maintain. If you do buy milk do not buy the big gallon jug unless you know that everyone in the house will drink it. 
\end{enumerate}
