
% Every good recipe starts with getting groceries. As a student, nobody has time for really long trips to the grocery store every week.

Below, I give a series of helpful tips and tricks that you can use in order to make sure that you are maximizing your efficiency when you cook. 

\begin{enumerate}
\item Get a solid Santoku knife. Really invest in a good one of these. These knives are effectively the Swiss-Army Knives of the kitchen. You can cut meat, veggies, pretty much anything with it. Make sure to get one that is of high quality and invest in re-sharpening if needed. 
\item Plan the week's recipes. This is actually key to making efficient trips to the store. Make sure that there are at least 4 meals on the menu for the week. It can be as simple as ``Make a lot of chicken and store in tupperware'', but it should at least be planned out. 
\item Invest in a single-serving blender. It makes breakfast shakes really easy, and can help make sauces for larger meals as well.
\item Invest in a single serving food processor. Unless you are a professional, chopping onions and garlic and peppers takes a lot of time. Roughly chop your veggies and then put them in the processor and pulse a couple times till you get the consistency you want. 
\item Buy quality meat and fish. This is just a rule since meat and fish are harder to cook and generally students buy less of them. So the little bit that you do buy, make sure that they are the good stuff. Frozen chicken is still a pretty good option. 
\item Don't ``overspice''. Spices are expensive and certainly have their place in food. However ``overspicing'' takes away the flavor of the ingredient itself and can really ruin a meal. Top culprits for ``overspicing'' are in grilled meats. 
\item Milk is tough to maintain. If you do buy milk do not buy the big gallon jug unless you know that everyone in the house will drink it. 
\item Prioritize your foods. When making food, try to keep in mind what would be good for lunch the next day. For instance, when making a vegetable stir-fry for dinner, leave a little left over so that you can then add chicken and some rice to the pan for a fried rice dish for lunch. 
\end{enumerate}
