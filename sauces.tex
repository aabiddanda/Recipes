
\recipe{Harissa}
\recipecolumn{
	\begin{itemize}
	\item 1 cup tomato paste
	\item 2 tablespoons ginger-garlic paste
	\item 1 tablespoon red chili flakes
	\item 1 teaspoon salt
	\item 1 teaspoon pepper
	\item 3 tablespoons lemon juice 
	\item $\frac{1}{4}$ cup roasted red pepper (optional)
	\end{itemize}
}{
	\begin{enumerate}
		\item Heat a dry pan to medium-high heat. Then place red chili, salt and pepper in pan and roast for 3 minutes. Allow to cool effectively.
		\item $^*$ If using roasted red pepper, make sure it is blended into a paste first $^*$
		\item In a separate bowl, mix ginger-garlic paste and tomato paste together. Add chili flake mix in and stir till mixed well. 
		\item Add lemon juice to paste and stir. Store in an airtight jar in the refridgerator until serving\footnote{Harissa is a condiment from Northern Africa, typically served with couscous or roasted meats}\footnote{This recipe is adapted from one that I found in a book entitled \textit{Healthy Cooking for people who don't have time to Cook} by Jeanne Jones}
	\end{enumerate}
}

\recipe{Charmoula}
\recipecolumn{
	\begin{itemize}
		\item 1 $\frac{1}{2}$ cups chopped cilantro
		\item 1 $\frac{1}{2}$ cups chopped parsley
		\item 2 tbsp. toasted and ground cumin seeds
		\item 2 cloves garlic
		\item Salt, to taste
		\item Black Pepper, to taste
		\item Turmeric, to taste
		\item $\frac{3}{4}$ Extra-virgin Olive Oil
		\item 2 tbsp. Lemon Juice
	\end{itemize}	
}{
	\begin{enumerate}
		\item blend all ingredients except for lemon juice in a food processor
		\item Add lemon juice and store in fridge till time to be used. Or store in an airtight jar until ready to serve alongside meat or vegetables.  \footnote{Chramoula is a cilantro-based marinade that is from Morocco and typically used for marinating meats or fish. This is a great marinade for salmon or chicken breast, but could be used for grilled vegetable skewers as well}
	\end{enumerate}
}

\recipe{Raita}
\recipecolumn{
	\begin{itemize}	
		\item $\frac{1}{2}$ cup diced\footnote{Or shredded} Cucumber
		\item $\frac{1}{2}$ cup diced tomato (optional)
		\item $\frac{1}{2}$ cup shredded carrot (optional)
		\item 1 tablespoon black pepper
		\item 2 tablespoons raita masala
		\item 3 cups greek yogurt
		\item 1 cup buttermilk
	\end{itemize}
}{
	\begin{enumerate}
		\item In a blender mix buttermilk, yogurt, raita masala, and pepper
		\item Add cucumber and tomato and store in airtight jar in refridgerator
	\end{enumerate}
}

\recipe{White Wine Sauce}
\recipecolumn{
	\begin{itemize}
		\item $\frac{1}{2}$ cup chicken stock or broth
		\item $\frac{1}{8}$ cup finely diced\footnote{Mashing shallots in a food processor with little chicken broth works great} shallots
		\item $\frac{1}{4}$ cup dry White Wine
		\item $3$ tablespoons butter
		\item $1$ tablespoon black pepper
	\end{itemize}
}{
	\begin{enumerate}
		\item Coat pan in cooking spray or small amount of olive oil, saute shallot for 2 minutes
		\item Increase heat slightly and add chicken stock and white wine
		\item After reduced to a quarter of previous volume, lessen heat and stir in butter \footnote{Recipe here is an adaptation from one found \href{http://www.myrecipes.com/recipe/white-wine-sauce}{here}}
	\end{enumerate}
}

\recipe{Herbed Garlic Butter}
\recipecolumn{
	\begin{itemize}
		\item 4 Large Garlic Cloves (Peeled)
		\item 1 Tbsp Salt
		\item 1 Tbsp Pepper
		\item 2 Tbsp Parsley (Chopped)
		\item $\frac{1}{2}$ stick of butter (unsalted)
	\end{itemize}
}{
	\begin{enumerate}
		\item Place the garlic cloves in a food processor and process till fine. 
		\item In separate bowl, heat butter for 30 seconds. Mix in all other ingredients in bowl and work into melted butter with spoon or fork till well-mixed. 
		\item Store in fridge to allow butter to cool to solid consistency.
	\end{enumerate}	
}
