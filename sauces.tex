
\recipe{Harissa}
\recipecolumn{
	\begin{itemize}
	\item 1 cup tomato paste
	\item 2 tablespoons ginger-garlic paste
	\item 1 tablespoon red chili flakes
	\item 1 teaspoon salt
	\item 1 teaspoon pepper
	\item 3 tablespoons lemon juice 
	\item $\frac{1}{4}$ cup roasted red pepper (optional)
	\end{itemize}
}{
	\begin{enumerate}
		\item Heat a dry pan to medium-high heat. Then place red chili, salt and pepper in pan and roast for 3 minutes. Allow to cool effectively.
		\item $^*$ If using roasted red pepper, make sure it is blended into a paste first $^*$
		\item In a separate bowl, mix ginger-garlic paste and tomato paste together. Add chili flake mix in and stir till mixed well. 
		\item Add lemon juice to paste and stir. Store in an airtight jar in the refridgerator until serving\footnote{Harissa is a condiment from Northern Africa, typically served with couscous or roasted meats}\footnote{This recipe is adapted from one that I found in a book entitled \textit{Healthy Cooking for people who don't have time to Cook} by Jeanne Jones}
	\end{enumerate}
}

\recipe{Charmoula}
\recipecolumn{
	\begin{itemize}
		\item 1 $\frac{1}{2}$ cups chopped cilantro
		\item 1 $\frac{1}{2}$ cups chopped parsley
		\item 2 tbsp. toasted and ground cumin seeds
		\item 2 cloves garlic
		\item Salt, to taste
		\item Black Pepper, to taste
		\item Turmeric, to taste
		\item $\frac{3}{4}$ Extra-virgin Olive Oil
		\item 2 tbsp. Lemon Juice
	\end{itemize}	
}{
	\begin{enumerate}
		\item blend all ingredients except for lemon juice in a food processor
		\item Add lemon juice and store in fridge till time to be used. Or store in an airtight jar until ready to serve alongside meat or vegetables.  \footnote{Chramoula is a cilantro-based marinade that is from Morocco and typically used for marinating meats or fish. This is a great marinade for salmon or chicken breast, but could be used for grilled vegetable skewers as well}
	\end{enumerate}
}

\recipe{Raita}
\recipecolumn{
	\begin{itemize}	
		\item $\frac{1}{2}$ cup diced\footnote{Or shredded} Cucumber
		\item $\frac{1}{2}$ cup diced tomato (optional)
		\item $\frac{1}{2}$ cup shredded carrot (optional)
		\item 1 tablespoon black pepper
		\item 2 tablespoons raita masala
		\item 3 cups greek yogurt
		\item 1 cup buttermilk
	\end{itemize}
}{
	\begin{enumerate}
		\item In a blender mix buttermilk, yogurt, raita masala, and pepper
		\item Add cucumber and tomato and store in airtight jar in refridgerator
	\end{enumerate}
}

\recipe{White Wine Sauce (Beurre Blanc)}
\recipecolumn{
	\begin{itemize}
		\item $\frac{1}{8}$ cup finely diced shallots
		\item $\frac{1}{2}$ cup dry White Wine \footnote{Chardonnay or Pinot Blanc work the best}
		\item $2$ sprigs of Thyme
		\item $6$ tablespoons butter
		\item $\frac{1}{4}$ cup heavy cream
		\item Capers (optional) 
	\end{itemize}
}{
	\begin{enumerate}
		\item Coat pan with 1 tablespoon of melted butter, and saute shallot for 5 minutes with a little bit of the white wine to release some of the initial flavor. 
		\item Increase heat to high and add white wine. Add Thyme sprigs as well.
		\item Once wine has reduced to approximately $\frac{1}{8}$ of its previous volume, constantly whisk (on medium heat now) the sauce while adding in 1 Tbsp of butter at a time.
		\item Add heavy cream and capers and salt to finish off the sauce. Optional : before this step and adding the capers, drain out the sauce so that there are no longer any shallots in the sauce (if you want a more refined sauce.)
		\item Serve with Salmon or Grilled Chicken and pasta.
	\end{enumerate}
}

\recipe{Herbed Garlic Butter}
\recipecolumn{
	\begin{itemize}
		\item 4 Large Garlic Cloves (Peeled)
		\item 1 Tbsp Salt
		\item 1 Tbsp Pepper
		\item 2 Tbsp Parsley (Chopped)
		\item $\frac{1}{2}$ stick of butter (unsalted)
	\end{itemize}
}{
	\begin{enumerate}
		\item Place the garlic cloves in a food processor and process till fine. 
		\item In separate bowl, heat butter for 30 seconds. Mix in all other ingredients in bowl and work into melted butter with spoon or fork till well-mixed. 
		\item Store in fridge to allow butter to cool to solid consistency.
	\end{enumerate}	
}

\recipe{Soy Ginger Marinade}
\recipecolumn{
	\begin{itemize}
		\item Honey
		\item Soy Sauce
		\item Ginger (chopped) 
		\item Garlic (minced)
	\end{itemize}
}{
	\begin{enumerate}
		\item Mix all ingredients together in bowl and pour over meat or veggies of choice.
		\item \textbf{Optional :} In pan, lightly fry ginger and garlic in small amount of oil and add oil to marinade as well.
	\end{enumerate}
}

\newpage

\recipe{Orange Balsamic Glaze}
\recipecolumn{
	\begin{itemize}
		\item Butter
		\item Shallots (minced)
		\item Brown Sugar
		\item Orange Juice
		\item Lemon Juice
		\item Balsamic Vinegar
		\item Salt
		\item Pepper
	\end{itemize}
}{
	\begin{enumerate}
		\item In a small pot or sauce pan, melt butter till it begins to lose its foam. Then place shallots and sautee for 2-3 minutes. 
		\item Mix together all other ingredients in the pot and alow to reduce while constantly stirring. 
		\item Strain out shallots (optional)
	\end{enumerate}
}


\recipe{Candied Sun-Dried Tomato Pesto}
\recipecolumn{
	\begin{itemize}
		\item Pine Nuts
		\item Sun-Dried Tomatoes
		\item Fresh Basil
		\item Parmesan Cheese
		\item Olive Oil
	\end{itemize}
}{
	\begin{enumerate}
		\item First begin by pulsing sun-dried tomatoes and pine nuts in a food processor with a little bit of olive oil. 
		\item Add in fresh basil and parmesan cheese into the food processor and process till fine paste.
	\end{enumerate}
}

\recipe{Salsa Verde}
\recipecolumn{
	\begin{itemize}
		\item Tomatillos 
		\item Lemon Juice 
		\item Red Onions 
		\item Lemon Zest
		\item Salt 
		\item Serrano Chilis (diced) 
	\end{itemize}
}{
	\begin{enumerate}
		\item Place all ingredients in a food processor and process till reasonably fine. 
	\end{enumerate}
}

